\partchapter{Aftermath, Something to Protect}{0}

\lettrine{A}{t} first Anna had been gratified to see the final Quidditch Cup go on so long—as a Gryffindor she was a bystander at the House Cup thing, it wasn’t like Gryffindor ever won. In contrast, last year’s Quidditch World Cup, to which her family had bought some very expensive tickets, had been over in \emph{ten minutes}, which was \emph{awful}. Modern Quidditch games had become too short, the Snitch caught much too quickly. It was a widely-talked problem among aficionados: broomstick enchantments had advanced, while the Snitch stayed the same regulation speed, with the result that Quidditch games had become shorter and shorter. At professional levels the sport of Quidditch had been reduced to a contest of who had the deepest pockets for their Seeker’s experimental racing broom, and the rest of the players might as well have been watching from the stands.

Everyone knew something had to be done, the situation had been getting worse for \emph{centuries} and now it was \emph{intolerable}. But the Confédération Internationale des Comités des Magiciens de Quidditch was mired in all the usual acrimony of the I.C.W., screaming disputes between Germans and Bulgarians, and somehow nobody could agree on \emph{exactly} how to fix the rules. To Anna the correct course seemed obvious, just make the Snitch fast enough to restore the four-hour or five-hour games of the early nineteenth century and the Golden Age of Quidditch. Except the Belgians thought the duration of a professional game should be two hours like in \emph{La Belle Époque} when Belgium had dominated Quidditch, and the lunatic Italians wanted to go back to the week-long Quidditch games of the fourteenth century, and Britain’s even crazier blood purists kept on talking up the occasional day-long Quidditch match as proof that broomsticks couldn’t \emph{really} have improved since everything was better in the old days \emph{which was not how the Interdict of Merlin worked.}

She was one hundred percent on the side of Harry Potter that it was time for Hogwarts to give up on those gibbering slow-coaches and just change the rules, starting here and now. But not by \emph{eliminating the Snitch,} that was going all the way back to \emph{eleventh century Kwidditch.} It didn’t matter if Headmistress Hufflepuff had first introduced the innovation because one of her students had wanted to play the game but not been suited to the usual roles. Snitches had caught on internationally because it was more exciting when the game could always end in the next minute.

Anna had been arguing this viewpoint at the top of her lungs for the last thirty minutes, quite forgetting to pay attention to the game. Thanks to a lucky coincidence of seating she’d been near the Boy-Who-Lived and his sign, and hence she’d managed to stake out her position right from the start.

She was aware, in the back of her mind, that if the Quidditch rules really \emph{did} change starting here and now, then this was the \emph{most important thing she’d ever do.} She could almost \emph{feel} the pressure of Time twisting around her as though the fate of Quidditch Itself were being settled this very day, and she was standing close to the centre of it…though she hadn’t got high-enough scores in Divination to actually sense anything like that, of course.

She hardly noticed when at one point the Boy-Who-Lived stood up to go to the bathroom.

The Boy-Who-Lived did catch her eye when he trudged back; Harry Potter looked a bit tired and wobbly, though his uniform appeared as trim as if he’d just changed into a new one.

She noticed half an hour later on, when Harry Potter seemed to sway a bit, and then hunch over, his hands going to cover up his forehead; it looked like he was prodding at his forehead scar. The thought made her slightly worried; everyone knew there was \emph{something going on} with Harry Potter, and if Potter’s scar was hurting him then it was possible that a sealed horror was about to burst out of his forehead and eat everyone. She dismissed that thought, though, and continued to explain Quidditch facts to the historically ignorant at the top of her lungs.

She definitely noticed when Harry Potter stood up, hands still on his forehead, and dropped his hands to reveal that his famous lightning-bolt scar was now blazing red and inflamed. It was \emph{bleeding}, with the blood dripping down Potter’s nose.

She stopped talking mid-sentence. Other people turned to look at what she was staring at.

“Professor McGonagall?” Harry Potter said in a wavering voice. There were tears in the corners of his eyes, which shocked her; the Boy-Who-Lived did \emph{not} seem like the sort of person who would burst into tears. Harry Potter raised his voice further, as though it were hard for him to speak. “Um, Professor McGonagall?”

Professor McGonagall turned away from where she was arguing with the Hufflepuff Quidditch team. The Head of Gryffindor’s eyes widened in shock, and then she was moving people out of her way, almost running. “Harry!” she said. “Your \emph{scar}!”

Silence was spreading, in a widening circle.

“I think,” Harry said, his voice still wavering but louder, “I think he’s back. I think I’m seeing—through Voldemort’s mind—”

Anna took a step back at You-Know-Who’s name and nearly fell over a bench. An older boy standing next to her gave a cry of dismay, and then the Boy-Who-Lived shrieked even louder.

“\scream{He’s killing them!}” screamed Harry Potter.

Half the Quidditch stadium turned to look at him.

“The ritual!” cried Harry Potter. “Blood of his servants! The blood, the life! He summoned them, he took their heads, their blood, the life, to renew his own—\scream{the Dark Lord rises, Voldemort is returned!}”

Madam Hooch blew a shrill whistle, and the Quidditch brooms that hadn’t already stopped in mid-air began to slow. For herself she wasn’t sure if this was a joke; if it was, Boy-Who-Lived or not, he was in more trouble than she could even imagine.

Professor McGonagall raised her wand into position for a Quieting Charm and Harry Potter caught her hand.

“Wait—” Harry Potter gasped, his voice lower, but still loud enough that she and the people near her could hear clearly. “He can be stopped—I see his mind, his mistake—he can be stopped \emph{now}—\shout{the way is still open! She’s following him! She whom Voldemort slew!}” Harry’s voice rose further, as Anna’s own mouth fell open in sudden confusion. “\scream{Return! Return, return, revive and stop him! Stop him, Hermione!}”

And then Harry Potter fell silent. He looked around at the people staring at him.

She’d just about decided that this had to all be a prank in \emph{unbelievably} poor taste, when a distant but sharp \emph{crack} filled the air.

Harry Potter swayed, and fell to his knees, even as her heart jumped into her throat. An explosion of excited babble rose around them.

She could still hear the words from Harry Potter’s mouth, as Professor McGonagall knelt next to him. “It worked,” Harry Potter gasped aloud, “she got him, he’s gone.”

“\emph{What?}” cried Professor McGonagall, then glanced around. “\emph{Quiet! Quiet, all of you!} Harry, what happened?”

Harry Potter was speaking rapidly but loudly. “Voldemort—tried to revive—he summoned Death Eaters \emph{and he killed them,} stole their blood and life—Hermione’s body was there, I don’t know why, maybe Voldemort was planning to use it for something—Voldemort came back, he resurrected himself, but Hermione \emph{followed him back} and she \emph{destroyed him,} he’s gone, it’s over. It happened in a graveyard near Hogwarts, it’s,” Harry Potter rose to his feet, still swaying, “I think it’s in that direction.” Harry Potter pointed in the rough direction the \emph{crack} had come from, “I’m not sure how far. The sound from there took twenty seconds to get here, so maybe two minutes on a broomstick—”

With a motion so smooth it looked unconscious, Professor McGonagall shifted into a stance and said “\emph{Expecto Patronum}.” She addressed the glowing cat that then appeared. “Go to Albus, tell him he must come at once—”

“Dumbledore’s gone!” cried Harry Potter. “The Headmaster is gone, Professor McGonagall! The Dark Lord trapped him, he reversed some kind of trap the Headmaster planned and Dumbledore was caught outside Time, he’s gone!”

The horrified babble around them rose in pitch.

“Go to Albus!” Professor McGonagall said to her Patronus.

The moonlit cat only looked at McGonagall sadly, and Anna sucked in her breath in sudden horror, feeling like someone had punched her in the stomach. It was real, it was all real, this wasn’t a joke.

“Professor McGonagall, Hermione is \emph{alive}!” Harry Potter raised his voice again. “She’s really alive and not an Inferius or anything, and she’s still there in the graveyard!”

“\emph{A broomstick!}” Professor McGonagall shouted. She turned to the players hovering motionless over the Quidditch field. “I need a broomstick. \shout{Now!}”

Despite everything, Anna raised a hand in mute protest, then caught herself, even as the Ravenclaw and Slytherin Seekers came zooming over (with excellent strategic sense, since they weren’t actually doing anything).

Harry Potter was already retrieving another broomstick from his pouch, a multi-person one.

Professor McGonagall saw this, and nodded firmly. “You stay here, Mr~Potter, unless there is some excellent reason you must be there. I will go at once.”

“You mustn’t!” squeaked Professor Flitwick, who’d shoved his tiny way through the crowd, occasionally running under someone’s legs. His eyes were wide, he looked as though he wanted to faint. “You have to stay at Hogwarts, Minerva! You—you’re the—” Professor Flitwick seemed to be having trouble speaking.

Professor McGonagall spun around to face Professor Flitwick, and then stopped, blood draining from her face.

Then she seized the broomstick from Harry Potter’s hand, and presented it to the tiny half-goblin Professor. “Filius,” she said crisply. All the incipient panic had disappeared from her voice, she now spoke in her crisp Scottish accent as though addressing lessons on Monday. “Look for the graveyard of which Mr~Potter spoke, find Miss~Granger. Apparate her to St. Mungo’s and then stay by her.”

“I think—” Harry Potter said hoarsely. “I think Transfiguration might have been used in combat there—Professor Quirrell tried to fight Voldemort—take precautions—”

Filius Flitwick nodded without halting in getting on the broomstick.

“Professor Quirrell’s dead!” wailed Harry Potter. The anguish in his voice carried clearly. “He’s dead! The Dark Lord killed him! His body—” Harry Potter choked up. “It’s there, in the graveyard.”

She stumbled back again, feeling it like another punch in her gut. Professor Quirrell had been—one of her favourite Professors, \emph{ever}, he’d made her rethink everything she’d believed about Slytherin, she’d known in some distant way that he was probably going to die very soon but to hear that he was really, truly dead…

The Boy-Who-Lived sat down on the bench, as if his legs couldn’t support him any more.

Professor McGonagall turned to the crowd, touching her wand to her throat. “\shout{Quidditch is over,}” her amplified voice boomed out. “\shout{Go back to your dormitories—}”

“\scream{Don’t!}” screamed Harry Potter.

Professor McGonagall turned to look at him.

Tears were leaking down the Boy-Who-Lived’s cheeks, he looked like the interruption had surprised himself as much as it had surprised anyone else. “It was Professor Quirrell’s last plot,” Harry Potter said, his voice breaking. The Boy-Who-Lived looked at the Quidditch players who had now flown to nearby, as though speaking to them directly. “His last plot.”

Harry Potter was floated off by Professor McGonagall to the infirmary. The other Professors ran off to oversee who-knew-what, leaving only Professors Sinistra and Hooch behind. At the stadium, rumours ran wild; Anna repeated everything she could remember hearing as best she could. Something had happened to Dumbledore, some Death Eaters had been summoned and killed (no, Harry Potter hadn’t said which ones), Professor Quirrell had gone out to face the Dark Lord and died for it, You-Know-Who had returned and died again, Professor Quirrell was dead, he was dead.

In time most of the students wandered off back to their dormitories, to sleep if they could.

Anna stayed in the stadium, and watched the rest of the game, ignoring her body’s need for sleep, and her eyes that often blurred with tears.

The Ravenclaw team put up a valiant fight.

But there was no Quidditch team anywhere that could’ve defeated the Slytherins that day.

Dawn was tingeing the sky when the Slytherins won their final game, the Quidditch Cup, and the House Cup.

%  LocalWords:  Confédération Internationale des Comités Magiciens Époque
%  LocalWords:  Kwidditch
