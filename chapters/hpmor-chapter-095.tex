\partchapter{Roles}{VI}

\section{The third meeting\\
(10:31\am, April 17th 1992)}

\lettrine{S}{pring} had begun, the late-morning air still crisp with the leavings of winter. Daffodils had bloomed amid the sprouting grass of the forest, the gentle yellow petals with their golden hearts dangling limply from their dead, greyed stems, wounded or killed by one of the sudden frosts that you often saw in April. In the Forbidden Forest there would be stranger life-forms, centaurs and unicorns at the least, and Harry had heard allegations of werewolves. Though from what Harry had read of real-life werewolves, that did not make the slightest bit of sense.

Harry didn’t venture anywhere near the border of the Forbidden Forest, since there was no reason to take the risk. He walked invisibly among the more ordinary life-forms of the permitted woods, wand in hand, a broomstick strapped to his back for easier access, just in case. He was not actually afraid; Harry thought it odd that he didn’t feel afraid. The state of constant vigilance, readiness for fight or flight, failed to feel burdensome or even abnormal.

On the edges of the permitted woods Harry walked, his feet never straying near the beaten path where he might be more easily found, never leaving sight of Hogwarts’s windows. Harry had set the alarm upon his mechanical watch to tell him when it was lunchtime, since he couldn’t actually look at his wrist, being invisible and all that. It raised the question of how his eyeglasses worked while he was wearing the Cloak. For that matter the Law of the Excluded Middle seemed to imply that either the rhodopsin complexes in his retina were absorbing photons and transducing them to neural spikes, or alternatively, those photons were going straight through his body and out the other side, but not both. It really did seem increasingly likely that invisibility cloaks let you see outward while being invisible yourself because, on some fundamental level, that was how the caster had—not \emph{wanted}—but \emph{implicitly believed}—that invisibility should work.

Whereupon you had to wonder whether anyone had tried Confunding or Legilimizing someone into implicitly and matter-of-factly believing that \emph{Fixus Everythingus} ought to be an easy first-year Charm, and then trying to invent it.

Or maybe find a worthy Muggle-born in a country that didn’t identify Muggle-born children, and tell them some extensive lies, fake up a surrounding story and corresponding evidence, so that, from the very beginning, they’d have a different idea of what magic could do. Though apparently they’d still have to learn a number of previous Charms before they became capable of inventing their own…

It might not work. Surely there’d been some organically insane wizards who’d truly believed in their own possibility of godhood, and yet had failed to become god. But even the insane had probably believed the ascension spell ought to be some grandiose dramatic ritual and not something you did with a carefully composed twitch of your wand and the incantation \emph{Becomus Goddus.}

Harry was already pretty sure it wouldn’t be that easy. But then the question was, \emph{why not}? What pattern had his brain learned? Could the reason be predicted in advance?

A slight fringe of apprehension crept through Harry then, a tinge of worry, as he contemplated this question. The nameless concern sharpened, grew greater—

\emph{Professor Quirrell?}

“Mr~Potter,” a soft voice called from behind him.

Harry spun, his hand going to the Time-Turner beneath his cloak; again the principle of being ready to flee upon an instant’s notice felt only ordinary.

Slowly, palms empty and turned outward, Professor Quirrell was walking towards him within the forests’ outskirts, coming from the general direction of the Hogwarts castle.

“Mr~Potter,” Professor Quirrell said again. “I know that you’re here. You know that I know that you’re here. I must speak to you.”

Still Harry said nothing. Professor Quirrell hadn’t actually said what this was about, and Harry’s sunlit morning walk about the forest edge had produced a mood of silence within him.

Professor Quirrell took a small step to the left, a step forward, another to the right. He tilted his head with a look of calculation, and then he walked almost directly towards where Harry stood, halted a few paces off with the sense of doom inflamed to the height of bearability.

“Are you still resolved upon your course?” Professor Quirrell said. “The same course you spoke of yesterday?”

Again Harry did not reply.

Professor Quirrell sighed. “There is much I have done for you,” the man said. “Whatever else you may wonder of me, you cannot deny that. I am calling in some of the debt. Talk to me, Mr~Potter.”

\emph{I don’t feel like doing this right now,} Harry thought; then: \emph{Oh, right.}

\later

Two hours later, after Harry had spun the Time-Turner once, noted down the exact time and memorized his exact location, spent another hour walking, went inside and told Professor McGonagall that he was currently talking to the Defence Professor in the woods outside Hogwarts (just in case anything happened to him), walked for a further hour, then returned to his original location exactly one hour after he’d left and spun the Time-Turner again—

\later

“What was that?” Professor Quirrell said, blinking. “Did you just—”

“Nothing important,” Harry said without pulling back the hood of his invisibility cloak, or taking his hand from his Time-Turner. “Yes, I’m still resolved. To be honest, I’m thinking I shouldn’t have said anything.”

Professor Quirrell inclined his head. “A sentiment which shall serve you well in life. Is there anything which is liable to change your mind?”

“Professor, if I already \emph{knew} about the existence of an argument which would change my decision—”

“True, for the likes of us. But you would be surprised how often someone knows what they are waiting to hear, yet must wait to hear it said.” Professor Quirrell shook his head. “To put this in your terms…there is a true fact, known to me but not to you, of which I would like to convince you, Mr~Potter.”

Harry’s eyebrows rose, though he realized in the next moment that Professor Quirrell couldn’t see it. “That’s in my terms, all right. Go ahead.”

“The intention you have formed is far more dangerous than you realize.”

Replying to this surprising statement did not take much thought on Harry’s part. “Define dangerous, and tell me what you think you know and how you think you know it.”

“Sometimes,” said Professor Quirrell, “telling someone about a danger can cause them to walk directly into it. I have no intention of having that happen this time. Do you expect me to tell you exactly what you must not do? Exactly why I am afraid?” The man shook his head. “If you were wizardborn, Mr~Potter, you would know to take it seriously, when a powerful magus tells you only to beware.”

It would have been a lie to say that Harry was not annoyed, but he also wasn’t an idiot; so Harry said merely, “Is there anything you \emph{can} tell me?”

Carefully, Professor Quirrell seated himself upon the grass, and took out his wand, his hand assuming a position that Harry recognized. Harry’s breath caught.

“This is the last time that I shall be able to do this for you,” Professor Quirrell said quietly. Then the man began to speak words that were strange, of no language Harry could recognize, intonation that seemed not quite human, words which seemed to slip from Harry’s memory even as he tried to grasp them, exiting from his mind as quickly as they entered.

The spell took effect more slowly, this time. The trees seemed to darken, branches and leaves staining, as though seen through perfect sunglasses that faded and attenuated light without distorting it. The blue bowl of the sky receded, the horizon which Harry’s brain falsely assigned a finite distance pulling back as it turned grey, and darker grey. The clouds became translucent, transparent, dissolving to let the darkness shine through.

The forest shaded, faded, abated into blackness.

The great sky river became visible once again, as Harry’s eyes adjusted, became able to see the largest object which human eyes could ever behold as more than a point, the surrounding Milky Way.

And the stars, piercingly bright and yet remote, out of a great depth.

Professor Quirrell breathed deeply. Then he raised his wand again (just barely visible, in the starlight without sun or moon) and tapped himself on the head with a sound like an egg cracking.

The Defence Professor also faded away, became likewise invisible.

A tiny disk of grass, illuminated by not much light at all, drifted unoccupied within empty space.

Neither of them spoke for a time. Harry was content to look at the stars, undistracted even by his own body. Whatever Professor Quirrell had called him here to say, it would be said in due time.

In due time, a voice spoke.

“There is no war here,” said a soft voice emanating from within the emptiness. “No conflict and battle, no politics and betrayal, no death and no life. That is all for the folly of men. The stars are above such foolishness, untouched by it. Here there is peace, and silence eternal. So I once thought.”

Harry turned to look at where the voice originated, and saw only stars.

“So you once thought?” Harry said, when no other words seemed to be forthcoming.

“There is nothing above the folly of men,” whispered the voice from the emptiness. “There is nothing beyond the destructive powers of sufficiently intelligent idiocy, not even the stars themselves. I went to a great deal of trouble to make a certain golden plaque last forever. I would not like to see it destroyed by human folly.”

Again Harry’s eyes reflexively darted toward where the voice should have been, again saw only emptiness. “I think I can reassure you on that score, Professor. Nuclear weapons don’t have a fireball extending out for…how far away is Pioneer 11? Somewhere around a billion kilometres, maybe? Muggles talk about nuclear weapons destroying the world, but what they actually mean is lightly warming up some of Earth’s surface. The \emph{Sun} is a giant fusion reaction and \emph{it} doesn’t vaporize distant space probes. The worst-case scenario for nuclear war wouldn’t even come close to destroying the Solar System, not that this is much of a consolation.”

“True while we speak of Muggles,” said the soft voice amid starlight. “But what do Muggles know of true power? It is not they who frighten me now. It is you.”

“Professor,” Harry said carefully, “while I have to admit I’ve rolled a few critical failures in my life, there’s a bit of distance between that and missing a saving throw so hard that the Pioneer 11 probe gets caught in the blast radius. There’s no realistic way to do that without blowing up the Sun. And before you ask, our Sun is a main-sequence G-type star, it \emph{can’t} explode. Any energy input would just increase the volume of the hydrogen plasma, the Sun doesn’t have a degenerate core that could be detonated. The Sun doesn’t have enough mass to go supernova, even at the end of its lifespan.”

“Such amazing things the Muggles have learned,” the other voice murmured. “How stars live, how they are preserved from death, how they die. And they never wonder if such knowledge might be dangerous.”

“In all frankness, Professor, that particular thought has never occurred to me either.”

“You are Muggle-born. I speak not of blood, I speak of how you spent your childhood years. There is a freedom of thought in that, true. But there is also wisdom in the caution of wizardkind. It has been three hundred and twenty-three years since the magical territories of Sicily were ruined by one man’s folly. Such incidents were more common in the years when Hogwarts was raised. Commoner still, in the time since Merlin. Of the time before Merlin, little remains to study.”

“There’s around thirty orders of magnitude of difference between that and blowing up the Sun,” Harry observed, then caught himself. “But that’s a pointless quibble, sorry, blowing up a country would also be bad, I agree. In any case, Professor, I don’t plan on doing anything like that.”

“Your choice is not required, Mr~Potter. If you had read more wizarding novels and fewer Muggle stories, you would know. In serious literature the wizard whose foolishness threatens to unleash the Shambling Bone-Men will not be deliberately bent on such a goal, that is for children’s books. This truly dangerous wizard shall perhaps be bent on some project of which he anticipates great renown, and the certain prospect of losing that renown and living out his life in obscurity will seem to him more vivid than the unknown prospect of destroying his country. Or he shall have promised success to one he cannot bear to disappoint. Perhaps he has children in debt. There is much literary wisdom in those stories. It is born of harsh experience and cities of ash. The most likely prospect for disaster is a powerful wizard who, for whatever reason, cannot bring himself to halt as warning signs appear. Though he may speak much and loudly of caution, he will not be able to bring himself actually to halt. I wonder, Mr~Potter, have you thought of trying anything which Hermione Granger herself would have told you not to do?”

“All \emph{right}, point taken,” said Harry. “Professor, I am well aware that if I save Hermione at the price of two other people’s lives, I’ve lost on total points from a utilitarian standpoint. I am \emph{extremely} aware that Hermione would not want me to risk destroying a whole country just to save her. That’s just common sense.”

“Child who destroys Dementors,” said that soft voice, “if it were only one country I feared you might ruin, I would be less concerned. I did not at first credit that your knowledge of Muggle science and Muggle practices would be a source of great power. I now credit it more. I am, in complete sincerity, concerned for the safety of that golden plaque.”

“Well, if science fiction has taught me anything,” said Harry, “it’s taught me that destroying the Solar System is not morally acceptable, especially if you do it before humanity has colonized any other star systems.”

“Then will you give up this—”

“No,” Harry said without even thinking before he opened his mouth. After a moment, he added, “But I do understand what you’re trying to tell me.”

Silence. The stars had not shifted, not even as they would have in an Earthly night sky, over time.

A very slight rustle, as of someone shifting their body. Harry realized that he had been standing for a while in the same position, and dropped down to the almost unseeable circle of grass that still stayed beneath him, careful not to touch the edges of the spell.

“Tell me this,” said the soft voice. “Why does that girl matter to you so much?”

“Because she is my friend.”

“In the English language as it is customarily used, Mr~Potter, the word ‘friend’ is not associated with a desperate effort to raise the dead. Are you under the impression that she is your true love, or some such?”

“Oh, not you too,” Harry said wearily. “Not you of all people, Professor. Fine, we’re best friends, but that’s \emph{all}, okay? That’s enough. Friends don’t let friends stay dead.”

“Ordinary folk do not do as much, for those they call friends.” The voice sounded more distant now, abstracted. “Not even for those they say they love. Their companions die, and they do not go in search of power to resurrect them.”

Harry couldn’t help himself. He looked over again, despite knowing it would be futile, and saw only more stars. “Let me guess, from this you deduce that…people don’t actually care as much about their friends as they pretend.”

A brief laugh. “They would scarcely pretend to care \emph{less}.”

“They care, Professor, and not just for their true loves. Soldiers throw themselves on grenades to save their friends, mothers run into burning houses to save their children. But if you’re a Muggle you don’t think there’s any such thing as magic to bring someone back to life. And normal wizards don’t…\emph{think outside the box} like that. I mean, most wizards aren’t searching for power to make \emph{themselves} immortal. Does that prove they don’t care about their own lives?”

“As you say, Mr~Potter. Certainly I myself would consider their lives pointless and without a shred of value. Perhaps, somewhere in their hidden hearts, they also believe that my opinion of them is the correct one.”

Harry shook his head, and then, in annoyance, cast back the hood of his Cloak, and shook his head again. “That seems like a rather \emph{contrived} view of the world, Professor,” said the dim-lit head of a boy, floating unsupported on a circle of dark grass amid stars. “Trying to invent a resurrection spell just isn’t something normal people would think of, so you can’t deduce anything from their not taking the option.”

A moment later, the dim-lit outline of a man sitting on the circle of grass was visible as well.

“If they \emph{truly} cared about their supposed loved ones,” the Defence Professor said softly, “they would think of it, would they not?”

“Brains don’t work that way. They don’t suddenly supercharge when the stakes go up—or when they do, it’s within hard limits. I couldn’t calculate the thousandth digit of pi if someone’s life depended on it.”

The dim-lit head inclined. “But there is another possible explanation, Mr~Potter. It is that people play the \emph{role} of friendship. They do just as much as that role requires of them, and no more. The thought occurs to me that perhaps the difference between you and them is not that you care more than they do. Why would you have been born with such unusually strong emotions of friendship, that you alone among wizardkind are driven to resurrect Hermione Granger after her death? No, the most likely difference is not that you care more. It is that, being a more logical creature than they, you alone have thought that playing the role of Friend would require this of you.”

Harry stared out at the stars. He would have been lying if he’d claimed not to be shaken. “That…can’t be true, Professor. I could name a dozen examples in Muggle novels of people driven to resurrect their dead friends. The authors of those stories clearly understood exactly how I feel about Hermione. Though you wouldn’t have read them, I guess…maybe Orpheus and Eurydice? I didn’t actually read that one but I know what’s in it.”

“Such tales are also told among wizardkind. There is the story of the Elric brothers. The tale of Dora Kent, who was protected by her son Saul. There is Ronald Mallett and his doomed challenge to Time. In Sicily before its fall, the drama of Precia Testarossa. In Nippon they tell of Akemi Homura and her lost love. What these stories have in common, Mr~Potter, is that they are all \emph{fiction}. Real-life wizards do not attempt the same, even though the notion is clearly \emph{not} beyond their imagination.”

“Because they don’t think they \emph{can}!” Harry’s voice rose.

“Shall we go and tell the good Professor McGonagall about your intention to find a way to resurrect Miss~Granger, and see what she thinks of it? Perhaps it has simply never occurred to her to consider that option…Ah, but you hesitate. You already know her answer, Mr~Potter. Do you know why you know it?” You could hear the cold smile in the voice. “A lovely technique, that. Thank you for teaching it to me.”

Harry was aware of the tension that had developed in his face, his words came out as though bitten off. “Professor McGonagall has not grown up with the Muggle concept of the increasing power of science, and nobody’s ever told her that when a friend’s life is at stake is a time when you need to \emph{think very rationally}—”

The Defence Professor’s voice was also rising. “The Transfiguration Professor is \emph{reading from a script}, Mr~Potter! That script calls for her to mourn and grieve, that all may know how much she cared. Ordinary people react poorly if you suggest that they go off-script. As you already knew!”

“That’s funny, I could have sworn I saw Professor McGonagall going off-script at dinner yesterday. If I saw her go off-script another ten times I might actually try to talk to her about resurrecting Hermione, but right now she’s new to that and needs practice. In the end, Professor, what you’re trying to explain away by calling love and friendship and everything else a lie is just \emph{human beings not knowing any better.}”

The Defence Professor’s voice rose in pitch. “If it were you who had been killed by that troll, it would not even \emph{occur} to Hermione Granger to do as you are doing for her! It would not occur to Draco Malfoy, nor to Neville Longbottom, nor to McGonagall or any of your precious friends! There is not one person in this world who would return to you the care that you are showing her! So \emph{why}? Why do it, Mr~Potter?” There was a strange, wild desperation in that voice. “Why be the only one in the world who goes to such lengths to keep up the pretence, when none of them will ever do the same for you?”

“I believe you are factually mistaken, Professor,” Harry returned evenly. “About a number of things, in fact. At the very least, your model of my emotions is flawed. Because you don’t understand me the tiniest bit, if you think that it would stop me if everything you said was \emph{true}. Everything in the world has to start somewhere, every event that happens has to happen for a first time. Life on Earth had to start with some little self-replicating molecule in a pool of mud. And if I were the first person in the world, no—”

Harry’s hand swept out, to indicate the terribly distant points of light.

“—if I were the first person in the \emph{universe} who ever really cared about someone else, which I’m \emph{not} by the way, then I’d be honoured to be that person, and I’d try to do it justice.”

There was a long silence.

“You truly do care about that girl,” the man’s dim outline said softly. “You care about her in the way that none of \emph{them} are capable of caring for their own lives, let alone each other.” The Defence Professor’s voice had become strange, filled with some indecipherable emotion. “I do not understand it, but I know the lengths you will go to because of it. You will challenge death itself, for her. Nothing will sway you from that.”

“I care enough to make an actual effort,” Harry said quietly. “Yes, that is correct.”

The starlight slowly began to fracture, the world shining through the cracks; slashes through the night showing tree trunks and leaves glowing in the sunlight. Harry raised a hand, blinking hard, as the returning brightness smashed into his dark-adjusted eyes; and his eyes automatically went to the Defence Professor, just in case an attack occurred while he was blinded.

When all the stars had gone and only daylight remained, Professor Quirrell was still sitting on the grass. “Well, Mr~Potter,” he said in his normal voice, “if that is so, then I shall give you what help I can, while I can.”

“You’ll \emph{what}?” Harry said involuntarily.

“My offer as I made it yesterday still stands. Ask and I will answer. Show me the same science books you deemed suitable for Mr~Malfoy, and I shall look them over and tell you what comes to mind. Don’t look so surprised, Mr~Potter, I would hardly leave you to your own devices.”

Harry stared, tear ducts still watering from the sudden light.

Professor Quirrell looked back at him. Something strange glinted in the pale eyes. “I have done what I can, and now I fear I must take my leave of you. Good—” and the Defence Professor hesitated. “Good day, Mr~Potter.”

“Good—” Harry began.

The man sitting on the grass fell over, his head impacting the ground with a light thud. At the same time the sense of doom diminished so sharply that Harry leapt to his feet, his heart suddenly in his throat.

But the figure on the ground slowly pushed back up to a crawling position. Turned to look at Harry, eyes empty, mouth slack. Tried to stand, fell back to the ground.

Harry took a step forward, sheer instinct telling him to offer a hand, although that was incorrect; the apprehension that rose up in him, however faint, spoke of continued danger.

But the fallen figure flinched away from Harry, and then slowly began crawl to away from him, in the general direction of the distant castle.

The boy standing amid the forest gazed after.

%  LocalWords:  pring Fixus Everythingus Becomus Goddus Elric Mallett Akemi
%  LocalWords:  Precia Testarossa Homura
